%!TEX TS-program = mypdflatex
%===========================================================================
%
\documentclass[12pt]{article}

% --------------------------------------------------
% Insira os seus dados e do exercício escolhido:
\def\discente{Paulo Henrique Santos Lima}
\def\matricula{201702779}
\def\aa{1}
\def\exercicio{1}
\def\myling{16} % Informe o número da linguagem selecionada.
% --------------------------------------------------

\usepackage[brazil]{babel}
\usepackage[T1]{fontenc}
\usepackage[utf8]{inputenc}
\usepackage[a4paper,top=1.5cm,bottom=1.5cm,left=2.0cm,right=1.5cm,nohead,nofoot]{geometry}
\usepackage{amsmath,amsfonts,amssymb}
\usepackage{enumitem}
\usepackage{calc}
\usepackage{xcolor}


\begin{document}
\subsection*{Atividade AA-\aa\ (exercício \exercicio)}
 \paragraph{\matricula -- \discente}
%
 \begin{itemize}
  \item \textcolor{blue}{$\mathcal{L}_\myling = \{w\in\Sigma^*\ = \{0,1\}^*\mid$ $\mid$w$\mid $ $ \geq$ 2 e w não contém 11$\}$.}
%
  \item  \textcolor{red}{Definição recursiva de $\mathcal{L}_\myling$:
%
  \begin{description}[labelwidth=\widthof{Recursão},labelindent=1.3\labelwidth,leftmargin=*,align=right]
   \item [Base:] $00,01 \in \mathcal{L}$.
   \item [Recursão:] Se $u\in \mathcal{L}$, então $u00,u01\in\mathcal{L}$.
   \item [Fecho:] Dada uma cadeia $u\in\Sigma^*$, $u\in\mathcal{L}_\myling$ se pode ser obtida a partir das cadeias básicas, com a aplicação da regra recursiva um número finito de vezes.
  \end{description}  
  }
% 
 \end{itemize}
\noindent\textbf{Obs.:} esta observação e os comandos de cor (``\verb|\textcolor{}{}|'') podem ser removidos da versão a ser submetida na plataforma Turing.
\end{document}
