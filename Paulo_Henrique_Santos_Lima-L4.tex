\documentclass[12pt]{article}

% --------------------------------------------------
% Insira os seus dados e do exercício escolhido:
\def\discente{Paulo Henrique Santos Lima}
\def\matricula{201702779}
\def\aa{2}
\def\myling{12} % Informe o número da linguagem selecionada.
% --------------------------------------------------

\usepackage[brazil]{babel}
\usepackage[T1]{fontenc}
\usepackage[utf8]{inputenc}
\usepackage[a4paper,top=1.5cm,bottom=1.5cm,left=2.0cm,right=1.5cm,nohead,nofoot]{geometry}

\usepackage{tikz}
\usetikzlibrary{automata,arrows,positioning}
\tikzset{
 node distance=2.5cm,
 initial text={$M_\myling$},
 double distance=1pt,
 every state/.style={semithick,fill=blue!20!white,minimum size=20pt,inner sep=0pt},
 every edge/.style={draw,->,>=stealth,auto,semithick,font=\ttfamily}
}


\begin{document}
\subsection*{Atividade AA-\aa}
 \paragraph{\matricula -- \discente}
%
 \begin{itemize}
  \item \textcolor{blue}{$\mathcal{L}_\myling = \{w\in\Sigma^*\ = \{0,1\}^*\mid$  w contém número par de ocorrências de 10$\}$.}
%
  \item  \textcolor{red}{Autômato finito determinístico que reconhece as cadeias $\mathcal{L}_\myling$:\\
  $M_\myling=\langle\Sigma=\{0,1\},S=\{s_0,s_1,s_2\},s_0,\delta,F=\{s_2\}\rangle$, com a função $\delta$ definida por:\\
  $$\begin{array}{|c|cc|}
   \hline
   \delta & 0   & 1\\
   \hline
      s_0 & s_1 & s_0\\
      s_1 & s_1 & s_2\\
      s_2 & s_2 & s_2\\
   \hline
  \end{array}$$}
 \end{itemize}
%
\begin{center}
 \begin{tikzpicture}[
  transform shape,
  scale=1.3
 ]
  \node[state, initial]                (s0) {$s_0$};
  \node[state, right of=s0]            (s1) {$s_1$};
  \node[state, accepting, right of=s1] (s2) {$s_2$};
%
  \draw (s0) edge[loop below]   node {1}   (s0)
        (s0) edge               node {0}   (s1)
        (s1) edge[loop below]   node {0}   (s1)
        (s2) edge[loop below]   node {0,1}   (s2)
        (s1) edge               node {1} (s2);
%             edge[out=330,in=280,looseness=1.2]
%                                node {$\varepsilon$} (s2);
 \end{tikzpicture}
\end{center}
%

\noindent\textbf{Obs.:} esta observação e os comandos de cor (``\verb|\textcolor{}{}|'') podem ser removidos da versão a ser submetida na plataforma Turing.
\end{document}
